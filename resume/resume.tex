\documentclass[line,margin]{res}
    \usepackage[none]{hyphenat}
    \usepackage{textcomp}
    \begin{document}
\name{{\LARGE Vincent Le}}
\address{vincentl@vt.edu | (703) 992-5598 | github.com/vinnyoodles}
\begin{resume}
	\vspace{-5mm}
	\section{EDUCATION}
	\textbf{Virginia Tech}, {\sl BS in Computer Science} GPA: 3.77\hfill May 2018\\Blacksburg, VA
	\section{EXPERIENCE}
	\textbf{Virginia Tech}, {\sl Undergraduate Teaching Assistant}\hfill Aug 2017 | Present\\Blacksburg, VA\\\begin{itemize} \itemsep 1pt
	\item Held office hours for the Data Structures and Algorithms course (CS 3114).
	\item Topics include sorting, graphs, trees, hashing and algorithm analysis.
	\end{itemize}
	\textbf{Mixmax}, {\sl Software Engineering Intern}\hfill Aug 2016 | Aug 2017\\San Francisco, CA\\\begin{itemize} \itemsep 1pt
	\item Implemented CRUD functions for supporting multiple Mixmax teams and organizations.
	\item Added Chrome extension logic to embed email draft metadata into Gmail HTML.
	\item Rewrote Backbone frontend view code into React reusable components.
	\item Added Chrome extension logic to embed Mixmax into Salesforce HTML for integrating with its metadata.
	\item Implemented a background sync of Salesforce contact and lead data for autofilling metadata in forms and templates.
	\end{itemize}
	\section{PROJECTS}
    \textbf{Collaboard}, {\sl Android, node.js, socket.io, MongoDB}\\ https://github.com/vinnyoodles/whiteboard\\\begin{itemize} \itemsep 1pt
	\item Published an Android app with realtime whiteboarding and voice communication.
	\item Wrote node.js server with WebSocket communication protocol using socket.io and deployed onto Heroku.
	\item Persisted canvas data with base64 encoding in MongoDB.
	\item Implemented audio communication by streaming bytes through WebSockets.
    \end{itemize}
	\textbf{cartera}, {\sl swift, xcode, UIKit, Parse}\\ https://github.com/makersofawesome/cartera\_ios\\\begin{itemize} \itemsep 1pt
	\item Developed an iOS app that allows peer to peer bank withdrawals.
	\item Wrote the networking client for the Capital One API and the Parse backend using NSURLConnection. 
	\item Implemented view logic following the MVC paradigm using UIKit and Mapkit.
	\end{itemize}
	\textbf{mosaic-layout}, {\sl swift, xcode, UIKit}\\ https://github.com/vinnyoodles/mosaic-layout\\iOS library implementation of a tile/mosaic based collection view, written with a protocol oriented design pattern.\\
	\section{SKILLS}
	\begin{tabular}{@{}ll}
		\textbf{Programming Languages}: & Java, C, Javascript, Swift   \\
		\textbf{Frameworks/Libraries}:  & node.js, backbone.js, React, UIKit, MapKit, Android  \\
	\end{tabular}
\end{resume}
\end{document}
