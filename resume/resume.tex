\documentclass[line,margin]{res}
\usepackage[none]{hyphenat}
\usepackage{textcomp}
\usepackage{hyperref} 
\hypersetup{
    colorlinks=true,
    urlcolor=blue,
}

\begin{document}
\name{{\LARGE Vincent Le}}
\address{vinnyoodles@gmail.com | \textbf{\href{https://github.com/vinnyoodles}{github.com/vinnyoodles}}}
\begin{resume}
	\vspace{-5mm}
	\section{EDUCATION}
		\textbf{Virginia Tech}, {\sl BS in Computer Science} \hfill May 2018\\Blacksburg, VA

	\section{EXPERIENCE}
		\textbf{Facebook}, {\sl Software Engineer}\hfill Jan 2020 | Present\\Seattle, WA\\
		\begin{itemize} \itemsep 1pt
			\item Built Marketplace features with a focus towards selling through the News Feed and Timeline
		\end{itemize}
		\textbf{Facebook}, {\sl Software Engineer}\hfill July 2018 | Dec 2019\\Menlo Park, CA\\
		\begin{itemize} \itemsep 1pt
			\item Worked in the Business Integrity team handling transparency and control around ads
			\item \textbf{\href{https://facebook.com/ads/library}{facebook.com/ads/library}}
			\item \textbf{\href{https://facebook.com/ads/library/report}{facebook.com/ads/library/report}}
		\end{itemize}
		\textbf{Virginia Tech}, {\sl Undergraduate Research Assistant}\hfill Feb 2018 | May 2018\\Blacksburg, VA\\
		\begin{itemize} \itemsep 1pt
			\item Developed bioinformatics application for running and managing genome analysis.
			\item Implemented backend using flask and frontend with Bootstrap.
			\item Managed background jobs using celery and redis.
			\item Used Docker Compose to manage MongoDB, flask, redis and celery instances.
		\end{itemize}

		\textbf{Virginia Tech}, {\sl Undergraduate Teaching Assistant}\hfill Aug 2017 | May 2018\\Blacksburg, VA\\
		\begin{itemize} \itemsep 1pt
		\item Held office hours for the Data Structures and Algorithms course (CS 3114) for students
			  to get help on projects and homeworks.
		\item Topics covered were sorting, searching, graphs, trees, hashing, recurrence relations and algorithm analysis.
		\end{itemize}

		\textbf{Mixmax}, {\sl Software Engineering Intern}\hfill Aug 2016 | Aug 2017\\San Francisco, CA\\
		\begin{itemize} \itemsep 1pt
			\item Implemented CRUD functions for supporting multiple Mixmax teams and organizations.
			\item Added Chrome extension logic to embed email draft metadata into Gmail HTML.
			\item Rewrote Backbone frontend view code into React reusable components.
			\item Added Chrome extension logic to embed Mixmax into Salesforce HTML for integrating with its metadata.
			\item Implemented a background sync of Salesforce contact and lead data for autofilling metadata in forms and templates.
		\end{itemize}


	\begin{tabular}{@{}ll}
		\textbf{Programming Languages}: & Java, Javascript, PHP   \\
		\textbf{Frameworks/Libraries}:  & node.js, React, React Native \\
	\end{tabular}
\end{resume}
\end{document}
